\section{Regular Language}

\subsection{DFA}
\begin{Definition}{DFA}{}
	A deterministic finite automata (DFA) is a 5-tuple $(Q, \Sigma, \delta, q_0, F)$, where
	\begin{itemize}
		\item $Q$ is a finite set called states
		\item $\Sigma$ is a finite set called alphabet
		\item $\delta: Q \times \Sigma \rightarrow Q$ is the transition function
		\item $q_0 \in Q$ is the start state (also called the initial state), and
		\item $F \subseteq Q$ is the set of accept states (final states).
	\end{itemize}
\end{Definition}

\subsection{NFA}
\begin{Definition}{NFA}{}
	A nondeterministic finite automaton is a $5$-tuple $\left(Q, \Sigma, \delta, q_0, F\right)$, where
	\begin{enumerate}
		\item $Q$ is a finite set of states,
		\item $\Sigma$ is a finite alphabet,
		\item $\delta: Q \times \Sigma_{\varepsilon} \longrightarrow \mathcal{P}(Q)$ is the transition function,
		\item $q_0 \in Q$ is the start state, and
		\item $F \subseteq Q$ is the set of accept states.
	\end{enumerate}
\end{Definition}

\subsection{Equivalence}
\begin{Definition}{$\epsilon$-closure}{}
	Let $N = (Q, \Sigma_{\epsilon}, \delta, q_0, F)$ be an NFA$_{\epsilon}$.
	The $\epsilon$-closure of a state $q$, denoted as $E(q)$ is:
	\[
		E(q) = \{ q \} \cup \delta(q, \epsilon) \cup \delta(\delta(q, \epsilon), \epsilon) \cup \cdots
	\]
	Let $R$ be a set of states for a given NFA. The $\epsilon$-closure of $R$ is
	\[
		E(R) = \bigcup_{q \in R} E(q)
	\]
\end{Definition}

\subsection{Properties}

\begin{Theorem}{Closure}{}
	Regular language is closed under union, concatenation, and star.
\end{Theorem}

\subsection{NFA-Regular Expression Equivalence}

\begin{Theorem}{}{}
	Regular expressions are equivalent to finite automata.
\end{Theorem}

\subsubsection{NFA to Regular Expression Conversion}
The conversion requires \textbf{generalized NFA}, which is a special type of NFA:
\begin{itemize}
	\item There is only one start state and only one final state.
	\item The start state is different from the final state.
	\item The transition function is $\delta: (Q \setminus \{ q_f \}) \times (Q \setminus \{ q_0 \}) \to \mathcal{R}$. In other words, the label of each transition is a regular expression (instead of a symbol for NFA).
	\item Suppose $\delta(a, b) = \mathcal{R}$, it means transition from $a$ to $b$ requires $\mathcal{R}$.
	\item If the generalized NFA has only two states - $q_0$ and $q_f$,
	      then the label on the transition is the regular expression for the language.
\end{itemize}
