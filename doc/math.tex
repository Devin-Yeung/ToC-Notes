\section{Math}

\begin{Def}{function}
	A function $f$ from $A$ to $B$ is a subset of $A \times B$ which satisfied:
	\begin{equation}
	\begin{aligned}
			&\forall x (x \in A \rightarrow \exists y (y \in B \land (x,y) \in f)) \\
			&(((x_1, y_1) \in B \land (x_1, y_2) \in B) \rightarrow y_1 = y_2)
	\end{aligned}
	\end{equation}	
\end{Def}

Noted that for an arbitrary $b \in B$(co-domain), there may not exist a corresponding $a \in A$(domain), but for every $a \in A$(domain), there must exist a corresponding $b \in B$(co-domain)

\begin{Def}{injection}
A function $f$ is said to be one-to-one, or an injection, if and only if $f (a) = f (b)$ implies that $a = b$ for all $a$ and $b$ in the domain of $f$. A function is said to be injective if it is one-to-one.
\[ ((f(a) = f(b)) \Rightarrow (a = b)) \Leftrightarrow f \; \text{is injection} \]
\end{Def}


\begin{Def}{surjection}
A function $f$ from $A$ to $B$ is called onto, or a \textbf{\textit{surjection}}, if and only if for every element $b \in B$ there is an element $a \in A$ with $f(a) = b$. A function $f$ is called surjective if it is onto.
\[ \forall b \in B \; \exists a \in A (f(a) = b) \Leftrightarrow f \; \text{is surjection} \]
\end{Def}


\begin{Def}{bijection}
The function $f$ is a one-to-one correspondence, or a \textbf{\textit{bijection}},
if it is both one-to-one and onto. We also say that such a function is bijective.
\end{Def}

