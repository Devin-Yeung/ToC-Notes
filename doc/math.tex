\section{Math}

\subsection{Function}

\begin{Definition}{function}
	A function $f$ from $A$ to $B$ is a subset of $A \times B$ which satisfied:
	\begin{equation}
		\begin{aligned}
			 & \forall x (x \in A \rightarrow \exists y (y \in B \land (x,y) \in f)) \\
			 & (((x_1, y_1) \in B \land (x_1, y_2) \in B) \rightarrow y_1 = y_2)
		\end{aligned}
	\end{equation}
\end{Definition}

Noted that for an arbitrary $b \in B$(co-domain), there may not exist a corresponding $a \in A$(domain), but for every $a \in A$(domain), there must exist a corresponding $b \in B$(co-domain)

\begin{Definition}{injection}
	A function $f$ is said to be one-to-one, or an injection, if and only if $f (a) = f (b)$ implies that $a = b$ for all $a$ and $b$ in the domain of $f$. A function is said to be injective if it is one-to-one.
	\[ ((f(a) = f(b)) \Rightarrow (a = b)) \Leftrightarrow f \; \text{is injection} \]
\end{Definition}


\begin{Definition}{surjection}
	A function $f$ from $A$ to $B$ is called onto, or a \textbf{\textit{surjection}}, if and only if for every element $b \in B$ there is an element $a \in A$ with $f(a) = b$. A function $f$ is called surjective if it is onto.
	\[ \forall b \in B \; \exists a \in A (f(a) = b) \Leftrightarrow f \; \text{is surjection} \]
\end{Definition}


\begin{Definition}{bijection}
	The function $f$ is a one-to-one correspondence, or a \textbf{\textit{bijection}},
	if it is both one-to-one and onto. We also say that such a function is bijective.
\end{Definition}


\subsection{Countability}

\begin{Definition}{}{}
	Given two sets $A$ and $B$, we say $A$ has the same cardinality as $B$ if there exists a bijection $f : A \to B$. This is usually denoted by $|A| = |B|$.
\end{Definition}

\begin{Definition}{Countable}{}
	A set $A$ is said to be {\bfseries \itshape countably infinite } if $|A| = |\mathbb{N}|$, and simply countable if $|A| \leqslant |\mathbb{N}|$
\end{Definition}

\begin{Example}{}{}
	Proof $A = \{ \text{all positive integers} \}$ is infinite countable
\end{Example}
Proof: Simply construct a bijection: $f: \mathbb{N} \to \mathbb{A}$ given by $f(n) = n + 1$.

\begin{Example}{}{}
	Proof $\mathbb{Z}$ is infinite countable
\end{Example}
Proof: Simply construct a bijection: $f: \mathbb{N} \to \mathbb{Z}$, where:
\[
	f(n) = \begin{dcases}
		f(1) = 0                              \\
		f(n) = \dfrac{n}{2}, n \text{ is odd} \\
		f(n) = -\dfrac{n - 1}{2}, n \text{ is odd}
	\end{dcases}
\]

\begin{Example}{}{}
	Proof $\mathbb{N} \times \mathbb{N}$ is infinite countable
\end{Example}
Proof: Construct a bijection using Cantor Pairing function,
which is a bijection: $f: \mathbb{N} \times \mathbb{N} \to \mathbb{N}$, where:
\[ f(x, y) = \dfrac{1}{2} (x + y)(x + y + 1) + y \]
without loss of generality, $\mathbb{N}^n$ is also infinite countable,
we can construct the bijection for $n > 2$
\[ \pi^{(n)} (k_{1},\ldots ,k_{n-1},k_{n}) = \pi (\pi^{(n-1)} (k_{1}, \ldots, k_{n-1}), k_{n}) \]